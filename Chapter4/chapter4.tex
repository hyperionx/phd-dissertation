%!TEX root = ../thesis.tex
%*******************************************************************************
%****************************** Fourth Chapter **********************************
%*******************************************************************************
\chapter{Current Transients in Memristive Devices}


\section[The Subthreshold Regime]{The Subthreshold Regime}

The initial primary focus of this thesis is on the characterisation of silicon-based memristors, which are the fundamental components of memristive systems. Since the discovery of the memristor and its importance to replicating synaptic activity had such a profound impact on the field of neuromorphic engineering, investigating additional nanoelectronic components and behaviours in this context will lead to new neuromorphic computing applications. \\

\noindent This chapter investigates a phenomenon known as the "current transient" that has yet to be deliberately applied to the demand of neuromorphic computing. The current transient phenomenon can be similarly represented by the current flowing through a defective capacitor in response to a step potential to produce rich dynamics, both growing and decreasing in conductance, and can be beneficial in a computational device. \\

\noindent This chapter begins by documenting and characterising the current transients based on available literature. The experimental procedures used throughout the chapter were then described, and strategies were developed to aid in the characterisation of current transients. This provides a deeper understanding of the physical models underpinning the transients, allowing for the further development of an integrative memristive system based on silicon oxide samples that are already available.

\subsection[Foundational Properties]{Foundational Properties}

Fundamentally, the processes of capacitive decay and dielectric relaxation are ideal to define a capacitor's response to a step voltage. Applying a constant voltage across its terminals causes the device current to decline until it ultimately comes to rest at a constant leakage current. This, however, is not always the case. When the voltage is applied for an extended period of time or at a high enough temperature, the current flowing through the device begins to grow as the oxide gets faulty and its resistance falls. This is known as oxide deterioration, an umbrella word for an oxide coating that becomes faulty over time as a result of environmental stress factors \cite{ghibaudo1999emerging}. \\

