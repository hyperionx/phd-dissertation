%!TEX root = ../thesis.tex
%*******************************************************************************
%****************************** Eighth Chapter **********************************
%*******************************************************************************

\chapter{Conclusion}

\section[Thesis Review]{Thesis Review}

This thesis has embarked on a comprehensive journey to advance biorealistic learning on memristive networks, systematically addressing the challenges and opportunities at the intersection of material science, device engineering, and neuromorphic computing. Each chapter has contributed distinct and impactful insights, collectively demonstrating the viability and potential of silicon oxide memristors for brain-inspired artificial intelligence.\\

\noindent Chapter 2: Theoretical Foundations served as the bedrock, providing essential primers in neuroscience and neuromorphic computing. Its primary impact lies in establishing a common language and conceptual framework for the interdisciplinary nature of this research. By detailing neuron models, synaptic plasticity mechanisms, and the principles of in-memory computing, this chapter ensured that subsequent discussions on device characteristics and system-level implementations were grounded in both biological plausibility and computational necessity. It highlighted the critical role of memristors in overcoming the Von Neumann bottleneck and laid the theoretical groundwork for understanding how these devices can emulate the fundamental learning rules of the brain.\\

\noindent Chapter 3: Fabrication and Characterisation Methodologies made a significant contribution by providing the empirical foundation for the entire thesis. The detailed description of the silicon oxide memristive device fabrication and electrical characterization yielded crucial insights into their resistive switching behaviors. The observation of unipolar and bipolar switching, multi-level states, and gradual conductance changes validated the suitability of these devices for analog synaptic emulation. The exploration of various conduction mechanisms and the proposal of a phenomenological model further deepened the understanding of the underlying physics, which is essential for reliable device design and performance prediction. This chapter's impact is in demonstrating the practical realization of memristive devices with properties conducive to neuromorphic applications.\\

\noindent Chapter 4: Current Transients in Memristive Devices presented a particularly impactful contribution by uncovering and characterizing the crucial phenomenon of current transients in silicon-based memristors operating in the subthreshold regime. The detailed documentation of their fundamental properties and tunability revealed a novel pathway for implementing biorealistic behaviors. The demonstration of combined potentiation and depression under the same voltage polarity, a highly desirable feature for simplifying neuromorphic circuit design, represents a significant advancement. This chapter's exploration of the physical implications of these transients and their relevance to homeostatic regulation directly informed the development of more biologically plausible learning mechanisms, marking a key step towards energy-efficient and adaptive SNNs.\\

\noindent Chapter 5: Neuromorphic Modelling Framework provided an indispensable contribution by translating experimental observations into a robust and predictive computational tool. The development of a SPICE-compatible compact model for resistance switching devices, specifically tailored to capture analog potentiation and depression, is a major achievement. This framework allows for accurate simulation of complex transient behaviors and enables circuit-level design and optimization before costly hardware fabrication. The insights gained from analyzing the impact of modified device stacks on transient characteristics further refined the understanding of the underlying physical mechanisms. The generalized SPICE model developed here is a powerful tool, impacting the ability to design and evaluate larger, more intricate neuromorphic circuits with confidence.\\

\noindent Chapter 6: Biorealistic Computing synthesized the findings from previous chapters into a compelling demonstration of memristive SNNs. This chapter's contribution lies in showcasing the practical application of these devices for brain-inspired computation, particularly in pattern recognition tasks. By delving into the simulation of hardware non-idealities and their impact on learning rules and training schemes, it highlighted the critical need for hardware-aware design. The chapter effectively demonstrated how the intrinsic analog dynamics of the memristors could be leveraged for efficient synaptic weight mapping, providing a tangible proof-of-concept for the computational performance of these systems in the presence of real-world imperfections. Its impact is in validating the functional capabilities of memristor-based SNNs.\\

\noindent Chapter 7: Homeostasis Optimisation addressed one of the most pressing challenges in analog neuromorphic hardware: maintaining robust network stability and mitigating programming variabilities. This chapter's key contribution is the introduction and investigation of homeostasis regularization techniques. By drawing inspiration from biological homeostatic mechanisms, it demonstrated how these principles can be integrated into memristive SNNs to enhance learning performance and ensure stable activity levels. The specific focus on biosignal processing and edge evaluation further underscored the practical relevance of these optimization strategies, impacting the reliability and deployability of memristive neuromorphic systems in real-world applications.\\

\noindent In conclusion, this thesis has made significant contributions across multiple layers of neuromorphic engineering. From the fundamental characterization of novel memristive devices and the elucidation of their transient behaviors, through the development of predictive modeling frameworks, to the practical implementation and optimization of biorealistic spiking neural networks, the work consistently bridges the gap between material science and artificial intelligence. The collective impact of these contributions lies in advancing the understanding and technological readiness of memristive devices for next-generation, energy-efficient, and brain-inspired computing, paving the way for truly autonomous and intelligent hardware at the edge.

\section[Future Works]{Future Works}

\noindent Building upon the current methodologies, the pursuit of truly in-situ (on-chip) training for Spiking Neural Networks utilizing memristive circuits represents a formidable yet imperative frontier in neuromorphic computing research. While the current ex-situ approach, leveraging detailed hardware-aware models for offline biorealistic learning and subsequent weight programming, yields high performance, it fundamentally separates the learning process from the operational hardware. The aspiration for in-situ learning is to imbue memristive SNNs with genuine autonomous learning capabilities, allowing them to adapt and learn from real-time data directly on the device, mirroring the adaptive nature of biological brains more closely. \\

\noindent Achieving in-situ biorealistic learning on memristive circuits necessitates overcoming a cascade of interconnected challenges across device fabrication, circuit design, and algorithmic implementation. From a device perspective, current memristor technologies, while promising, still contend with significant issues that hinder their deployment in complex, self-learning systems. Device-to-device variability, cycle-to-cycle conductance fluctuations, limited endurance, and issues with multi-level programming precision remain critical hurdles. Future work must concentrate on advanced material engineering and sophisticated fabrication processes to mitigate these inherent non-idealities. This includes exploring novel material stacks that exhibit more linear and symmetric conductance modulation, improving the uniformity of resistive switching mechanisms, and developing scalable manufacturing techniques that ensure high yield and long-term reliability for large-scale memristor arrays. Furthermore, research into memristive devices that inherently possess differentiable switching characteristics or those whose physics can be directly leveraged for gradient computation would be transformative, moving beyond mere synaptic emulation to active participation in the learning process.\\

\noindent At the circuit level, the direct translation of biorealistic learning into an analog memristive substrate is particularly challenging. The requirement to perform a backward pass, which typically involves transposing weight matrices and propagating error signals through a network, poses substantial design complexities. Future research must devise innovative analog circuit architectures capable of accurately computing and propagating these error signals. This could involve exploring bidirectional memristor devices or developing specialized peripheral circuitry that facilitates the reverse flow of information, possibly through time-multiplexing techniques, without incurring prohibitive area or power overheads. Moreover, implementing the non-differentiable spiking nonlinearity of neurons (the core challenge for biorealistic learning in SNNs) directly in analog hardware while still enabling gradient estimation requires novel neuron circuit designs that incorporate hardware-friendly surrogate gradient functions. These analog neuron circuits must be robust to noise and thermal variations, providing reliable spiking behavior and a usable "pseudo-derivative" for on-chip learning. The energy and memory demands for storing intermediate neuron activations across time for biorealistic learning also represent a formidable hurdle, pushing the boundaries of on-chip memory integration and low-power analog storage solutions.\\

\noindent From an algorithmic standpoint, while the core biorealistic learning framework remains powerful, its adaptation for in-situ operation will likely involve more biologically plausible approximations. This includes investigating alternative local learning rules that are simpler to implement on-chip but can still approximate global error signals, such as variants of Equilibrium Propagation or Direct Feedback Alignment, which minimize the need for exact backpropagation. The interplay between these simplified algorithms and the inherent noise and non-idealities of the memristive devices themselves must be carefully studied; it is conceivable that certain forms of device stochasticity could even be harnessed as a beneficial component of the learning process, rather than merely being a challenge to overcome. Ultimately, the successful realization of in-situ biorealistic learning on memristive circuits will demand a tightly coupled co-design approach, where advancements in device physics, circuit engineering, and neuromorphic algorithms are iteratively developed and optimized in unison, forging a path toward truly autonomous and energy-efficient intelligent hardware.\\