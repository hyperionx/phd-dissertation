%!TEX root = ../thesis.tex
%*******************************************************************************
%****************************** Eighth Chapter **********************************
%*******************************************************************************

\chapter{Conclusion}

\section[Thesis Review]{Thesis Review}

\section[Future Works]{Future Works}

\noindent Building upon the current methodologies, the pursuit of truly in-situ (on-chip) training for Spiking Neural Networks utilizing memristive circuits represents a formidable yet imperative frontier in neuromorphic computing research. While the current ex-situ approach, leveraging detailed hardware-aware models for offline Backpropagation Through Time (BPTT) and subsequent weight programming, yields high performance, it fundamentally separates the learning process from the operational hardware. The aspiration for in-situ learning is to imbue memristive SNNs with genuine autonomous learning capabilities, allowing them to adapt and learn from real-time data directly on the device, mirroring the adaptive nature of biological brains more closely. \\

\noindent Achieving in-situ BPTT on memristive circuits necessitates overcoming a cascade of interconnected challenges across device fabrication, circuit design, and algorithmic implementation. From a device perspective, current memristor technologies, while promising, still contend with significant issues that hinder their deployment in complex, self-learning systems. Device-to-device variability, cycle-to-cycle conductance fluctuations, limited endurance, and issues with multi-level programming precision remain critical hurdles. Future work must concentrate on advanced material engineering and sophisticated fabrication processes to mitigate these inherent non-idealities. This includes exploring novel material stacks that exhibit more linear and symmetric conductance modulation, improving the uniformity of resistive switching mechanisms, and developing scalable manufacturing techniques that ensure high yield and long-term reliability for large-scale memristor arrays. Furthermore, research into memristive devices that inherently possess differentiable switching characteristics or those whose physics can be directly leveraged for gradient computation would be transformative, moving beyond mere synaptic emulation to active participation in the learning process.\\

\noindent At the circuit level, the direct translation of BPTT into an analog memristive substrate is particularly challenging. The requirement to perform a backward pass, which typically involves transposing weight matrices and propagating error signals through a network, poses substantial design complexities. Future research must devise innovative analog circuit architectures capable of accurately computing and propagating these error signals. This could involve exploring bidirectional memristor devices or developing specialized peripheral circuitry that facilitates the reverse flow of information, possibly through time-multiplexing techniques, without incurring prohibitive area or power overheads. Moreover, implementing the non-differentiable spiking nonlinearity of neurons (the core challenge for BPTT in SNNs) directly in analog hardware while still enabling gradient estimation requires novel neuron circuit designs that incorporate hardware-friendly surrogate gradient functions. These analog neuron circuits must be robust to noise and thermal variations, providing reliable spiking behavior and a usable "pseudo-derivative" for on-chip learning. The energy and memory demands for storing intermediate neuron activations across time for BPTT also represent a formidable hurdle, pushing the boundaries of on-chip memory integration and low-power analog storage solutions.\\

\noindent From an algorithmic standpoint, while the core BPTT framework remains powerful, its adaptation for in-situ operation will likely involve more biologically plausible approximations. This includes investigating alternative local learning rules that are simpler to implement on-chip but can still approximate global error signals, such as variants of Equilibrium Propagation or Direct Feedback Alignment, which minimize the need for exact backpropagation. The interplay between these simplified algorithms and the inherent noise and non-idealities of the memristive devices themselves must be carefully studied; it is conceivable that certain forms of device stochasticity could even be harnessed as a beneficial component of the learning process, rather than merely being a challenge to overcome. Ultimately, the successful realization of in-situ BPTT on memristive circuits will demand a tightly coupled co-design approach, where advancements in device physics, circuit engineering, and neuromorphic algorithms are iteratively developed and optimized in unison, forging a path toward truly autonomous and energy-efficient intelligent hardware.\\