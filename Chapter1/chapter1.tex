%!TEX root = ../thesis.tex
%*******************************************************************************
%*********************************** First Chapter *****************************
%*******************************************************************************

\chapter{Introduction}  %Title of the First Chapter

\section[Goals and Motivations]{Goals and Motivations}

Neuromorphic computing aims to bridge the gap between neuroscience and artificial intelligence by emulating the structural and functional principles of biological neural systems. At the heart of this vision is a pressing research question: how can we implement biorealistic learning on memristive networks to develop energy-efficient, scalable and adaptive computing architectures? Addressing this question requires a multidisciplinary approach, combining insights from neurobiology, electronic materials science and computational modelling. \\

\noindent Traditional von Neumann architectures \cite{von1993first}, characterised by their separation of memory and processing units, struggle to efficiently handle tasks that the human brain performs effortlessly, such as pattern recognition, sensory integration, and decision making. In contrast, the brain achieves these feats with remarkable energy efficiency and adaptability, in part due to the tightly coupled nature of computation and memory in its neural circuits. Mimicking these biological characteristics in silicon and emerging nanotechnologies has become the guiding principle of neuromorphic engineering \cite{saighi2015plasticity}. \\

\noindent Recent advances in memristive technologies have reignited interest in neuromorphic computing. Memristors, or memory resistors, are two-terminal non-volatile devices that can emulate synaptic plasticity by adjusting their conductance based on the history of voltage and current applied. This property lend itself to naturally support learning rules such as Hebbian learning \cite{hebb2005organization} and spike-timing-dependent plasticity (STDP) \cite{markram1997regulation}. When arranged in crossbar arrays, memristors offer a promising platform for in-memory computation, which can significantly reduce the power and latency associated with traditional data transfer bottlenecks. \\

\noindent This work presents a comprehensive discussion of biorealistic learning mechanisms and their physical realisation on memristive networks. By grounding the discussion in the neuroscientific principles that underlie learning and cognition, the research aims to elucidate how these biological processes can be abstracted and implemented in hardware. \\

\noindent The work begins with an overview of the biological basis of computation, providing an essential neuroscience primer. It then moves to device-level considerations, discussing the properties of memristive devices and their integration into neuromorphic architectures. Throughout, the emphasis is on aligning computational models with biological fidelity, while navigating the constraints and opportunities offered by emerging nanotechnologies.

\section[Thesis Overview]{Thesis Overview}

Chapter 1: Introduction provides comprehensive motivations into biorealistic learning on memristive networks, bridging device-level physics with system-level neuromorphic computation. The work adopts a layered approach, systematically progressing from fundamental neuroscience principles and device characterization to advanced modeling and the implementation of brain-inspired learning algorithms. This overview provides a clear and cohesive narrative of the thesis, highlighting the progression of the research from foundational concepts to practical applications and future directions.\\

\noindent Chapter 2: Theoretical Foundations lays the groundwork by providing essential primers in neuroscience and neuromorphic computing. It begins with a detailed overview of neuron anatomy, electrophysiology, and spiking neuron dynamics, including models such as the Hodgkin-Huxley and Leaky Integrate-and-Fire (LIF) neurons. The chapter then delves into the fundamentals of memristive devices, their unique properties, and their role in in-memory computing paradigms. Special attention is given to how memristors encode synaptic plasticity, such as Spike-Timing Dependent Plasticity (STDP) and Hebbian learning, and how these devices integrate into hierarchical modular architectures and hardware-software co-design strategies for neuromorphic systems.\\

\noindent Chapter 3: Fabrication and Characterisation Methodologies details the experimental procedures and results for the silicon oxide memristive devices central to this research. It describes the metal-insulator-metal (MIM) device structure, the radio frequency magnetron sputtering fabrication process, and the electrical characterization methodologies employed. The chapter presents the observed unipolar and bipolar switching modes, multi-level switching, and gradual conductance changes, providing a thorough understanding of the device's fundamental electrical characteristics. It also explores various conduction mechanisms in silicon oxide, proposing a phenomenological model to explain the observed resistive switching activities.\\

\noindent Chapter 4: Current Transients in Memristive Devices focuses on a unique phenomenon: current transients in silicon-based memristors operating in the subthreshold regime. This chapter documents the fundamental properties of these transients, their tunability through device stressing and pulse amplitude modulation, and their potential for neuromorphic behaviors such as combined potentiation and depression within the same voltage polarity. It explores the physical implications of these transients, discussing conflicting theories (ionic vs. electronic currents) and their relevance to homeostatic regulation in spiking neural networks.\\

\noindent Chapter 5: Neuromorphic Modelling Framework introduces a comprehensive empirical modeling framework for the resistance switching devices, specifically addressing the analog potentiation and depression observed in the subthreshold regime. This chapter details the development of a SPICE-compatible compact model that accurately captures the complex transient behaviors. It discusses the impact of modified device stacks (e.g., ITO vs. gold contacts) on transient characteristics, leading to insights into the underlying physical mechanisms. The chapter culminates in a generalized SPICE model that can predict device behavior across a range of operating voltages, serving as a crucial tool for advanced neuromorphic circuit simulations.\\

\noindent Chapter 6: Biorealistic Computing delves into the application of these memristive devices in Spiking Neural Networks (SNNs) for biorealistic computing. It establishes the brain-like analogies that motivate SNN design and discusses the unique aspects of the spiking paradigm compared to traditional deep neural networks. The chapter explores the simulation of non-idealities inherent in memristive devices and their impact on learning rules and training schemes. It demonstrates how the intrinsic dynamics of the memristors, particularly their analog conductance modulation, can be mapped to synaptic weights for inference and classification tasks, showcasing the computational performance of these hardware-aware SNNs.\\

\noindent Chapter 7: Homeostasis Optimisation addresses critical challenges in implementing biorealistic learning on analog hardware, particularly programming variabilities and the need for robust network stability. This chapter explores optimization motivations and introduces homeostasis regularization techniques to mitigate the effects of device non-idealities. It investigates how homeostatic mechanisms, inspired by biological systems, can be integrated into memristive SNNs to maintain stable activity levels and enhance learning performance, especially in the context of biosignal processing and edge evaluation.\\

\noindent Chapter 8: Conclusion provides a comprehensive review of the thesis, summarizing the key contributions and findings across all chapters. It reiterates how the research bridges device physics and system-level neuromorphic computation, demonstrating the viability of memristive silicon oxide devices for brain-inspired computing. Finally, the chapter outlines promising avenues for future work, including further experimental validation, exploration of advanced circuit architectures, and the development of more biologically plausible learning algorithms, paving the way for truly autonomous and energy-efficient intelligent hardware.\\